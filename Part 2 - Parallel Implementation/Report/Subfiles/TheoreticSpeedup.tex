\section{Theoretic Speed-up}
In this section we see what the theoretic speed-up should be of the parallel implementation. We do this for part 1 of our problem, as part 2 is difficult to predict due to a loop where the number of operations is difficult to determine. We further split part 1 into its different functions to see how long each should take. This is done by checking how many operations need to be computed by the Titan Black, which uses compute capability 3.5 and has 15 multiprocessors.

\subsection{\textit{populate\_sin\_cos\_maps}}
Here, each thread computes a sine and a cosine function and we have 512 threads. This means that this function should take:

\[\frac{N/A}{M} = \frac{512 * 2 / 32} {15} = 2.133 \quad ClockCycles\]

Where 
\begin{itemize}
	\item N = Number of times operation O needs to execute
	\item A = Amount of times operation O can be performed per clock cycle per multi-processor
	\item M = Number of multi-processors
\end{itemize}

\subsection{\textit{calculate\_points}}
Here we have to perform 3 bit shifts (if we consider multiplication of an integer by 2 also as a bit shift), 2 integer addition, 5 float additions and 4 floating point multiplication for every 2 points.

\[ \frac{((3/644) + (2/160) + (5/192) + (4/192))2^{n-1}}{15} \]
\[=4.26889*10^{-3}*2^{n-1} \quad ClockCycles\]

\subsection{\textit{calculateMin} and \textit{calculateMax}}
These 2 functions have the same operations and are executed 3 times in total for as many times as we have points. Here we have 3 integer additions and 1 multiplication, 3 bit shifts and 1 comparison, hence we get:

\[3\frac{3/160 + 1/32 + 3/644 + 1/160}{15}2^{n}\]
\[=3*0.00406056*2^n \quad ClockCycles\]

\subsection{\textit{map\_points\_to\_pixels}}
This last function has 2 float multiplications and 2 float additions per points, meaning:
\[\frac{2/192 + 2/192}{15}2^{n}\]
\[=\frac{1}{720}2^n \quad ClockCycles\]